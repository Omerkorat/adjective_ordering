\documentclass[12pt]{article}
\usepackage[hmargin={1in},vmargin={1in,1in},foot={.6in}]{geometry}   
\geometry{letterpaper}              
\usepackage{color,graphicx}
\usepackage{setspace}
\usepackage{amsmath}
\usepackage{amssymb}
\usepackage{varioref}
\usepackage{textcomp}
\usepackage{textcomp}
\usepackage{mflogo}
\usepackage{wasysym}
\usepackage[normalem]{ulem}
\usepackage{hyperref}

\newcommand{\HRule}{\rule{\linewidth}{0.25mm}}

\usepackage{fancyhdr} % This should be set AFTER setting up the page geometry
\pagestyle{plain} % options: empty , plain , fancy
\lhead{}\chead{}\rhead{}
\renewcommand{\headrulewidth}{.5pt}
\lfoot{}\cfoot{\thepage}\rfoot{}
\newcommand{\txtp}{\textipa}
\renewcommand{\rm}{\textrm}
\newcommand{\sem}[1]{\mbox{$[\![$#1$]\!]$}}
\newcommand{\lam}{$\lambda$}
\newcommand{\lan}{$\langle$}
\newcommand{\ran}{$\rangle$}
\newcommand{\type}[1]{\ensuremath{\left \langle #1 \right \rangle }}

\newcommand{\bex}{\begin{exe}}
\newcommand{\eex}{\end{exe}}
\newcommand{\bit}{\begin{itemize}}
\newcommand{\eit}{\end{itemize}}
\newcommand{\ben}{\begin{enumerate}}
\newcommand{\een}{\end{enumerate}}

\newcommand{\gcs}[1]{\textcolor{blue}{[gcs: #1]}}
\definecolor{Green}{RGB}{10,200,100}
\newcommand{\ndg}[1]{\textcolor{Green}{[ndg: #1]}}
\newcommand{\jd}[1]{\textcolor{red}{[jd: #1]}}

\thispagestyle{empty}

\begin{document}

{\flushright

\vspace{25pt}
Irvine, California\\[20pt]

\noindent XXX DATE\\[25pt]}


\noindent Dear Dr.~S\ae b\o,\\

\noindent We would like to thank you and the two anonymous reviewers for your reviews and comments on our paper, ``On the grammatical source of adjective ordering preferences.'' As you will recall, you highlighted the following three concerns, which you encouraged us to consider in the development of a new submission to \emph{Semantics and Pragmatics}: 

\ben

\item \emph{The most important issue we see is with the tension between hierarchical and
	linear structure. While linear structure is usually held to be relevant for
	incremental processing, your argument rests on hierarchical structure, and
	although this argument seems to be oriented towards incremental processing
	(``speakers employ the most useful, least subjective adjectives early in this
	semantic process"), you do not take note of the mismatch between linear and
	hierarchical structure in prenominal languages like English. Modification
	may proceed outward from the noun in semantic composition as a theoretical
	model, but it is doubtful if it (only) does so in processing, and processing
	is arguably what is relevant for pressures from successful reference
	resolution and utility for effective communication. Both reviewers emphasize
	this dilemma, and Reviewer A points to relevant work in the area of
	incremental interpretation of adjectives.}

	We have highlighted the tension between hierarchical and linear structure in our revision in an attempt to foreground the puzzle that adjective ordering presents: here is a case where hierarchical, compositional structure appears to take precedence over linear, incremental processing. The work on predictive looks during incremental processing (i.e., the Sedivy eye-tracking study and related work hinted at by Reviewer A) serves to increase the interest of this tension. However, the early uptake of semantic information evidenced by predictive looking in eye-tracking studies does not rule out that the semantic composition of nominal phrases proceeds outward from the noun; it is this semantic composition process that stands to explain the role of subjectivity in adjective ordering preferences. When discussing timing in our proposed account, we are careful to limit the discussion to the timing of semantic composition: adjectives closer to the noun compose with the noun semantically earlier than nouns that are farther away. In our revision, we discuss in more detail the potential differences between the timing of this semantic process and the predictive inferences listeners make when processing language, a discussion which clarifies our position while further motivating its novelty.
	
\item \emph{Second, a connection is made between subjectivity and uncertainty, or
	liability to alignment errors (``noise"), which is not warranted without
	further argument. On the one hand, as Reviewer A notes, it may not follow
	from every theory of subjective meaning that subjectivity leads to
	uncertainty. And on the other hand, subjectivity is not the only property
	that may lead to uncertainty, or a high risk of misalignment: vague
	adjectives and relative adjectives are also likely to introduce noise. In
	fact, the adjective ``small", which you seem to use as an example of
	subjective adjectives, is usually held to be both relative and vague, and
	whether it is independently subjective is a matter of debate (see, e.g.,
	Kennedy 2016).}

	We have clarified what we mean by ``subjectivity'' in our revision, namely the operationalization of the concept from Scontras, Degen, and Goodman (2017): an adjective is subjective to the extent that it admits faultless disagreement between two speakers about whether some object might hold the property the adjective names. We now acknowledge that many factors could contribute to the possibility for faultless disagreement, including vagueness and relativeness. We also acknowledge various formal accounts of subjectivity. However, given that our focus is on a specific behavioral operationalization of subjectivity, we believe fully exploring all of these various factors would take us far afield of the relevant point in our argument: the potential for faultless disagreement can lead to alignment errors between speakers and listeners when they do not agree on the application of a property to a set of objects.
	
\item \emph{And third, the particular analysis you propose in terms of noise and
	subsective, non-intersective modification is not sufficiently explicit to
	show what it is supposed to show, namely, that it pays for reference
	resolution to start with less noisy modifiers. The definition (3) is not
	immediately intelligible, and it does not become clear exactly how
	subsectivity leads to non-commutativity. (In fact, not all theories of
	adjective meaning predict that adjectives like ``small" are not intersective;
	according to Kennedy (2007), for example, the relativity is built into the
	context-sensitive standard of comparison function that comes with the
	positive morpheme.) One would need to see a step-by-step demonstration that
	``small brown box" indeed results in fewer errors, or a higher probability of
	intended referent retrieval, than ``brown small box".}

	We have provided the requested demonstration of how ordering adjectives with respect to subjectivity results in fewer alignment errors, thereby increasing the probability of the listener retrieving the intended referent. 
	
\een
In the remainder of this letter, we consider in more detail each of the reviewers' concerns, which we have attempted to address while adhering to the length guidelines for Remarks and Replies.

Thank you again for the thorough and thoughtful comments on our work. We hope that you will like the new version of the paper. Please let us know if you require additional information. We look forward to hearing from you!\\[25pt]


\noindent Yours sincerely,\\[10pt]

\noindent Gregory Scontras, Judith Degen, and Noah D.~Goodman

\newpage




\subsubsection*{Reviewer A}

\ben

\item \emph{Is there evidence that tells us whether referential processing is
sensitive to heirarchical vs.~linear structure?  I do not know the
processing literature well, but I do know of one fairly prominent study
(with many follow-ups) that does not appear to work in the author's favor. 
This is the work on ``referential contrast effects" originally done by Julie
Sedivy (using the Visual World Paradigm), and subsequently by lots of other
people.  What this work shows is that when subjects are attending to a
visual context in which there is a ``target" object that satisfies both a
noun and an adjective meaning (e.g., a tall glass) and ``competitor" object
that satisfies the adjective but not the noun (e.g., a tall pitcher),
subjects fixate on the target more quickly when there is also a ``contrast"
object that satisfies the noun but not the adjective (e.g. a small cup).  In
particular, they fixate on the target in the contrast condition before they
have processed the noun.  This result seems difficult to explain if the
subjects are engaging in referential processing according to heirarchical
order, in the way that the reasoning in the ms.~expects, rather than
according to linear order, in the way that would appear to lead to exactly
the opposite predictions about adjective ordering universals, at least for
head-final languages, though it would make the correct predictions for
head-initial languages.  (On the other hand, it would not predict the
mirror-image ordering preferences that we actually see.)}

As mentioned in our response to the editor's first point above, in our revision we have highlighted the tension between hierarchical vs.~linear processing. It bears noting that this tension is precisely the reason why we chose to present this work as Remarks \& Replies: we discuss a robust empirical finding, consider possible accounts, and settle on a new account that raises questions for the research community about the nature of incremental reference resolution as it relates to hierarchical semantic composition. In an attempt to more faithfully represent the psycholinguistic findings on incremental reference resolution, we know discussion the Sedivy et al.~study, as well as its predecessor in Eberhard et al. We also discuss recent results from Qing et al.~that call into question the interpretation of predictive looks in visual-world eye-tracking studies as clear evidence of reference resolution. From our vantage point, the picture of incremental \emph{semantic} processing is an unsettled one; rather than claiming to solve the issue, we present our proposal in service of added nuance. 


\item \emph{The author ends up operationalizing the effects of subjectivity on
reference selection in terms of uncertainty.  This is interesting, but
deserves a bit more discussion.  First, how does this relate to actual
theories of uncertainty?  Would a relativist theory like Lasersohn's lead us
to expect this result, for example? Lasersohn is actually very explicit to
reject an account of subjectivity based on uncertainty of meaning. And at
least from one perspective (so to speak), it seems that a relativist theory
would NOT predict greater uncertainty of meaning for subjective predicates: 
if there is a convention for autocentric interpretation, then there is
little uncertainty about the meaning of a subjective predicate as uttered by
S:  it picks out the set of things that satisfy the predicate from S's
perspective.  On the other hand, from a different perspective, maybe a
relativist theory would give rise to such a prediction:  even if there is
such a convention, there is going to be uncertainty for any hearer H about
what the actual extension of the predicate is, given that H doesn't share
S's perspective.  But even if we grant all that, how do these essentially
pragmatic expectations about language use get cashed out and worked into the
grammar in the form of adjective ordering preferences?}

\emph{There is actually one theory of subjectivity that explicitly links
subjectivity to uncertainty of meaning, and even hypothesizes that
(something like) different degrees of uncertainty correspond to different
degrees of subjecitvity.  This is the approach advocated by Kennedy and
Willer (2016) (SALT paper ``Subjective attitudes and counterstance
contingency").  The author has clearly tried to be agnostic about the
``right" account of subjectivity in this paper, but if the proposals make
more sense when they're embedded in one approach compared to another, that
would constitute a theoretically interesting result, which should be pointed
out.}

As mentioned in our response to the editor's second point above, in our revision we have taken care to clarify that by ``subjectivity'' we intend the operationalization of the concept by Scontras et al.~(2017). However, in an attempt at thoroughness, we now explicitly mention some of the many factors that could contribute to perceived subjectivity, as well as to formal accounts of the subjective vs.~objective distinction. Given the narrow aims of our short paper, we leave it to future work to fully vet these accounts in light of our proposal.


\item \emph{It would be good to show exactly how subsective modification breaks the chain of commutativity.  The paper points to one example --- comparison
class-dependent vague predicates --- but not all subjective adjectives show
the profile of such expressions.  And it is not even clear that such
expressions are not intersective from an actual compositional point of view.
If the adjective is context-dependent relative to threshold of application
(as is usually thought), then `small elephant' and its kin can be
interpreted intersectively provided the threshold of `small' is fixed in the
right way.  True, there's a kind of dependency between the adjective and the
noun, but this need not be a compositional one.}

As mentioned in our response to the editor's third point above, in our revision we have added detail to the demonstration of how ordering with respect to subjectivity serves successful reference resolution. We break this discussion down into two notions. The first is minimizing errors in alignment where a listener might misclassify potential referents according to a subjective property; we now walk through the math that demonstrates how subjectivity ordering minimizes these alignment errors. The second notion concerns correctly retrieving the intended referent; we now walk through how dependence on a modified nominal when fixing the interpretation of a modifier can amplify misclassifications and thus decrease the probability of correctly classifying the intended referent---which can be mitigated by subjectivity-based ordering.


\item \emph{The discussion of ``referential utility" on p.~9 needs to be cleaned up. I get the general point here, which is that, assuming that modification
isn't vacuous, the set of objects that each adjective gets to ``make a
decision about" monotonically decreases as we move out from the noun. 
(Sticking to the case of intersective modification!)  But the discussion
makes it sound as the inner adjectives also necessarily exclude more objects
than the outer ones.  This is of course not true, and is contingent on the
facts.  If I have a set of 100 boxes, 99 of which are cardboard and 50 of
which are big, then in `big cardboard box', `big' provides greater
information gain than `cardboard'.}

We have clarified our discussion of the reference-establishing potential of adjectives. We now state explicitly that we measure this potential in terms of the number of possible referents considered (and not the number of possible referents excluded).


\item \emph{Returning to the issues of processing above:  the following poster was
presented at CUNY a couple years ago:}

\emph{Redford, R., \& Chambers, C.G. (March 2016). The Good, the Bad, and the Ugly: Incremental Interpretation of Evaluative Adjectives. Poster presented at the
CUNY Conference on Human Sentence Processing, Gainesville, FL.}

\emph{If I remember correctly, it had the interesting result that evaluative
modifiers were less likely to give rise to a referenetial contrast effect in
a Sedivy-style referential processing study. This is also relevant to the
current proposal, because it suggests (and I think this is what the authors
of the poster proposed) that evaluative (subjective) adjectives are more
likely to be non-restrictive than restrictive. What does this mean in the
context of the current proposal?  Does it make sense, if the author is
correct that greater subjectivity leads to less referential utility?  Or
does it provide another explanation for order effects, given the well-known
structural constraints on restrictive vs.~non-restrictive interpretations
(e.g.~Larson 1998, SALT paper ``Events and modification in nominals")?  Or
could the latter be another result of the former?  }

We have been unable to track down the relevant reference mentioned by the reviewer, and so we have refrained from mentioning it in our revision. \gcs{unless we do track it down and want to mention it} Still, if the findings are correctly recalled, they would be very much in line with our proposal: establishing nominal reference seems to be the purview of adjectives that make their semantic contributions early. To the extent that one views Larson's proposal in terms of restrictive vs.~non-restrictive modification, our proposal would be in line with his observations as well. However, it isn't clear to us that the primary distinction to be drawn from Larson's work is one of restrictiveness.

\een


\subsubsection*{Reviewer B}

\ben

\item \emph{One issue that keeps on coming up in my mind has to do with the
incrementality of language interpretation. We know from a large body of work
that language interpretation is incremental and that comprehenders will
start to make predictions about an intended referent even before a sentence
is over.  Thus, from this angle, a communicatively maximally efficient
approach might be to say the most informative, reliable, helpful adjectives
*first*, to help the listener narrow down the potential set of referents as
fast as possible.  Under this view, it seems odd that subjective adjectives
would occur BEFORE more objective adjectives in prenominal languages like
English. The situation is, of course, unproblematic for languages with
post-nominal adjectives since there, more objective adjectives *do* linearly
precede more subjective adjectives. Perhaps the ordering constraints in
prenominal languages are due to some grammatical constraint – if so, it
would be helpful to say something about this, to be more explicit about why
prenominal adjective languages would mention the `less useful'
adjectives *first*.}  

%\emph{I may be misunderstanding or overlooking something very fundamental in the paper, but basically, the question of ``why not always make sure that more helpful adjectives occur linearly before less helpful adjectives'' keeps up coming up for me. In other words, if the right account is a communicatively-oriented, reference-based account, it's not yet fully clear to me how it works for languages with prenominal adjectives where more subjective adjectives linearly precede less subjective adjectives.  If I am overlooking something critical, perhaps the authors could make that part more explicit.}

We share the reviewer's concern and puzzlement. As mentioned in our response to the editor's point 1 and Reviewer A's point 1, we view this puzzle as a contribution of our proposal, rather than as a bug. In our revision, we have highlighted the tension between incremental language processing and semantic composition while highlighting potential pitfalls in interpreting evidence of the former as informative for the latter.

\item \emph{As a side note, I was also wondering, typologically, how the number of
prenominal adjective languages compares to the number of postnominal
adjective languages?   (especially since, at least on my understanding, the
authors' analysis/account/story is simpler/more straightforward for
postnominal adjective languages)}

A quick look at the World Atlas of Language Structures (WALS) suggests that post-nominal adjective languages are about twice as common as pre-nominal languages like English (878 vs.~373). However, we hesitate to overinterpret this result, given the opportunistic nature of the typological data.

\item \emph{A question related to the point above has to do with whose needs are driving the adjective ordering process, as well as what assumptions are being made
about language production. On page 10, the authors state that ``Thus,
speakers employ the most useful, least subjective adjectives early in this
semantic process where there is the greatest potential for misalignment''.
There seems to be some tension here between a speaker-oriented process and
an addressee-oriented process. On the one hand, the authors convincingly
articulate how less subjective adjectives are more helpful for communication
(addressee-oriented need). At the same time, though, talking about speakers
employing the most useful adjectives early in the process – where
`early' means early in a bottom-up, noun-based combination process (?)
– seems to be a speaker-oriented claim which could be seen as being at
odds with the incremental nature of language comprehension that I mention in
the preceding point (at least for prenominal adjective languages like
English).   It would be helpful if the authors could clarify their
assumptions about the speaker- vs. addressee-oriented nature of the process
as well as what they mean by `early'  – since (if I understood
correctly) it is early from the linguistic construction-of-noun-phrases
point of view but *not* early from the addressee's `when do I hear the
adjective in the linear string in English' point of view.}

We are careful to state that our temporal claims relate to semantic composition, or the linguistic construction-of-the-noun-phrase. The issue of whose needs drive the ordering process is an interesting one. It isn't clear to us that it is possible to neatly divide the speaker's goal's from those of the listener. It could well be that the speaker's utility metric considers whether the listener successfully retrieves the intended referent; the listener's utility metric is also likely to include this consideration. Thus, ordering with respect to subjectivity in service of successful reference resolution serves both the speaker's and the listener's goals. \gcs{Judith, do you want to add anything about speaker vs.~listener goals?} As to the issue of incrementality, please see our response to Reviewer B's point 1, Reviewer A's point 1, and the editor's point 1. 


\item \emph{The title is ``on the grammatical source of adjective ordering
preferences'' but seems to me that the authors ultimately argue for a
relatively non-grammatical and more communicatively-oriented source.}

Our account relies primarily on the hierarchical nature of semantic composition, which we take to be a feature of the grammar. Our title is meant to reflect this fact.

\item \emph{p.2 ``has proven elusive, owing to the complex, careful empirical work
required to test these hypotheses'' – this sounds a bit negatively
inclined towards the prior work, as if they were not capable of conducting
careful empirical work. Is this intended?}

The only systematic behavioral study of adjective ordering among the work we mention comes from Martin 1969, which is why we highlight empirical work as an obstacle to scientific progress in this domain. In our revision, we have removed the word ``careful'' to avoid unintended negative connotations. %\jd{\emph{also, i removed the bit that said there hasn't been any corpus work, because we do cite wulff, who of course has done corpus work}}

\item \emph{p.2 second paragraph: hypotheses $=>$ hypothesis?}

We have corrected this error.

\item \emph{Section 2, p.2-3: the authors use the term `extremely' a few times when talking about Scontras (``an extremely strong correlation'' and ``finding
extremely high correlations').  I would recommend leaving out the modifier
`extremely' and just saying strong correlation (and perhaps including a
number indicating correlation strength) – using a word like
`extremely' without numerical support struck me as a bit odd.}

We have provided the numerical support (i.e., $r^2$ values and bootstrapped 95\% confidence intervals) for use of the word `extremely.'

\item \emph{Top of p.4: Given how much attention the cartographic approach has received in prior literature, it seems to me that it was perhaps dismissed a little too fast. As a related side note, one could claim that gradient judgements
could emerge from a categorical grammatical system, e.g.~via interaction with
a `performance' factor or through an interplay of factors. Someone might
argue that the grammar has a rigid structure but that extra-grammatical
factors interact with that and hence the outcome is gradient. I'm not
saying I personally would argue for this but I can see how someone could
make this claim.}

The cartographic approach offers descriptive generalizations about adjective ordering but leaves unexplained \emph{why} lexical semantic classes should be ordered in the way that they are. We therefore dismiss the cartographic approach as a viable strategy for explaining the observed orderings.

\item \emph{p.4: the link between (i) being more inherent to the object and (ii) being a more salient/accessible property was not entirely clear to me.  Can't we have properties that are very inherent (at least on some understanding of the term) but not very salient when we think about the object? E.g.~an inherent property of cats is that they are warm-blooded but I don't think this is necessarily a very salient/accessible property, when compared to properties like `furry'. But I may be misunderstanding what is meant by `inherent' in the work being discussed in this section, so maybe that could simply be clarified?  (This is not an issue for the authors' claims in this paper, obviously.)}

We agree that the issue of inherentness is a rather nebulous one. As far as we can tell, inherentness deals with those properties that spring to mind when considering some noun, such that recognizing an object means recognizing its inherent properties as well. Thus, to the extent that being warm-blooded is a property we take to be inherent to cats, thinking of a cat leads to thinking about the warm-blooded property, which would increase its activation and therefore its accessibility.

\item \emph{p.10: the example about the tacky polyester shirt is very interesting and I think the the question of ``whether speaker goals influence ordering preferences'' is a really important question to answer if one wants to argue for a communication-based account such as this one.  So I might recommend that the authors not `dismiss' this question so fast and perhaps spend more time on this issue.}

While we agree that this issue is an interesting one, we hesitate to pursue it further without the relevant empirical work. However, in our revision, we have more clearly highlighted the possibility that subjectivity might play less of a role in ordering preferences when there isn't pressure from successful reference resolution.


\item \emph{p.10-11:  I was very glad to see discussion of acquisition work. That seems like another intriguing direction for future work.}

We wholeheartedly agree!

\een

\end{document}















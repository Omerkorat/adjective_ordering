\documentclass[12pt]{article}
\usepackage[hmargin={1in},vmargin={1in,1in},foot={.6in}]{geometry}   
\geometry{letterpaper}              
%\usepackage[parfill]{parskip}
\usepackage{color,graphicx}
\usepackage{setspace}
\usepackage{amsmath}
\usepackage{amssymb}
\usepackage{varioref}
\usepackage{textcomp}
%\usepackage{avm}
\usepackage{textcomp}
\usepackage{mflogo}
\usepackage{wasysym}
\usepackage[normalem]{ulem}
\usepackage{hyperref}

\newcommand{\HRule}{\rule{\linewidth}{0.25mm}}

\usepackage{fancyhdr} % This should be set AFTER setting up the page geometry
\pagestyle{plain} % options: empty , plain , fancy
\lhead{}\chead{}\rhead{}
\renewcommand{\headrulewidth}{.5pt}
\lfoot{}\cfoot{\thepage}\rfoot{}
\newcommand{\txtp}{\textipa}
\renewcommand{\rm}{\textrm}
\newcommand{\sem}[1]{\mbox{$[\![$#1$]\!]$}}
\newcommand{\lam}{$\lambda$}
\newcommand{\lan}{$\langle$}
\newcommand{\ran}{$\rangle$}
\newcommand{\type}[1]{\ensuremath{\left \langle #1 \right \rangle }}

\newcommand{\bex}{\begin{exe}}
\newcommand{\eex}{\end{exe}}
\newcommand{\bit}{\begin{itemize}}
\newcommand{\eit}{\end{itemize}}
\newcommand{\ben}{\begin{enumerate}}
\newcommand{\een}{\end{enumerate}}

%\linespread{1.5}
\thispagestyle{plain}

\begin{document}

{\flushright

\vspace{25pt}
Gregory Scontras\\
Judith Degen\\
Noah D.~Goodman\\
Department of Psychology\\
Stanford University\\
Stanford, CA 94305\\[20pt]

\noindent November 19, 2015\\[20pt]}


\noindent Dear Editor,\\

\noindent We would like to thank you and the two anonymous reviewers for your helpful comments on our paper, ``Subjectivity predicts adjective ordering preferences.'' As you will recall, you and Reviewer 2 were in agreement that adjective ordering is a topic of broad interest, which would be an excellent topic for a PNAS paper; yet you expressed concern that our work does not move the inquiry forward far enough. We believe our results represent a significant step forward for the study of ordering preferences. Put simply, we believe our results document an important empirical regularity, which will drive future theoretical development, and our methods bring new empirical rigor to the long-running study of ordering preferences. 

Adjective ordering preferences continue to recur in discussions of language universals precisely because of their reported regularity within and across languages. However, even in the best studied case---English---the investigations to date have been largely impressionistic. Most authors report their own intuitions or the intuitions of a handful of informants, using a highly constrained set of adjectives and nouns; these data are sufficient for suggesting a potential phenomenon, but we believe insufficient for fully understanding it. The only large-scale empirical studies of ordering preferences are Martin 1969, who limits himself to behavioral measures (i.e., speaker intuitions), and Wulff 2013, who limits herself to corpus measures (i.e., relative frequencies). With respect to ordering preferences, our study marries these two approaches, using both behavioral measures and corpus analyses to arrive at clear estimates of the preferences themselves---showing that there are reliable \emph{quantitative} phenomena to be explained.

There is even less empirical work on the factors that contribute to these ordering preferences, which is where our paper makes its largest contribution. Most authors do not attempt to operationalize their hypotheses, let alone test them. Once again, Martin 1969 stands apart as the sole exception. Of the four different aspects of adjective meaning that he hypothesized would predict ordering preferences, the best-performing measure was adjective ``definiteness;'' it accounts for 55\% of the variance in his preference data. Our study has much wider validity: we use more adjectives, 10 times more nouns, and two different operationalizations of subjectivity. We show that subjectivity predicts 88\% of the variance. %Put simply, no other study comes close in its empirical rigor or e. 
No other study comes close in empirical rigor, nor in success at predicting ordering preferences.

We do not claim to have a theoretical explanation of why subjectivity should predict ordering preferences (as was noted by both reviewers), but we do not believe this decreases the significance of the finding. 
A correlation of $0.94$ between two \emph{a priori} unrelated but theoretically interesting behavioral measures is itself an important finding. In this case, we believe it is one that will serve as a clarion call for new theoretical work.
(Indeed, this finding has already led to long discussions with a half-dozen leading experts in language and cognition.)
%While we agree with Reviewer 1 that the finding that subjectivity predicts ordering preferences lacks an explanation, we disagree on its significance: our finding represents serious progress in the investigation of ordering preferences. 

With clear estimates of both the preferences themselves and an aspect of adjective meaning that determines them (i.e.~subjectivity), work on ordering preferences may finally move beyond informal descriptions to rigorous scientific inquiry. 
%
Moreover, the implications of this work are not limited to the relative order of adjectives. The most obvious extension is adverb ordering, which appears similarly robust across languages and similarly poorly understood. In our revision, we emphasize the significance of our findings and their potential extension to other cross-linguistic regularities.

In the remainder of this letter, we consider in more detail each of the reviewers' concerns.


\subsubsection*{Reviewer 1:}

\ben

\item \emph{The fundamental factor in predicting adjective ordering is whether an adjective is used to form a complex concept/subkind description or not.}

We find this hypothesis intriguing---perhaps concept-formability indeed determines ordering preferences (and therefore correlates with subjectivity)?
%We considered it unlikely that a binary distinction like concept formation would be able to predict the gradience in our preference data. Still, we found the hypothesis very compelling, which is why 
We set out to test the hypothesis; as with the studies in our paper, the work lies in operationalizing an abstract notion like whether or not an adjective tends to form a complex concept. The literature on the topic (McNally and Boleda, 2004; Svenonius, 2008) presupposes that intuitions about concept formability are systematic and generalizable; the closest we found in these papers to a proposal for an empirical measure of this factor is the following distinction.

According to McNally and Boleda, the key issue is one of entailment. When an adjective modifies a noun intersectively, the objects described hold both the property named by the noun and the property named by the adjective: a ``male architect'' is both male and an architect (McNally and Boleda, 2004:179, ex.~2). When an adjective and a noun combine to form a complex concept (i.e., a subkind description), the objects described hold the property named by the noun, but not necessarily the property named by the adjective; the modification is (ostensibly) subsective. The authors give the Catalan example \emph{arquitecte t\`{e}cnic} ``technical architect,'' which names architects but not necessarily technical things (McNally and Boleda, 2004:179, ex.~1; cf.~the discussion of \emph{wild rice} in Svenonius 2008). 
%
Using our original set of materials, we tested whether the objects named by an adjective-noun description hold 1) the property named by the adjective, and 2) the property named by the noun.\footnote{The full experiment is \href{http://web.stanford.edu/~scontras/adjective_ordering/experiments/9-concept-formability/concept-formability.html}{viewable online here}.} We tested 40 participants on Mechanical Turk. 


The semantic analysis given by these authors to adjectives that form complex concepts requires them to compose first with nouns, before run-of-the-mill intersective adjectives; thus, the fundamental factor in predicting adjective ordering ought to be whether an adjective forms a complex concept. Does concept-formability predict ordering preferences?
The adjective concept-formability ratings predict 8\% of the variance in our preference data (r$^{2}=0.08$; 95\% CI [0.00,  0.33]). The noun ratings predict 36\% of the variance (r$^{2}=0.36$; 95\% CI [0.07,  0.62]). Recall that at its worst, subjectivity predicts 70\% of the variance in our preference data.
While it is quite possible that concept-formability plays an important role for some cases (such as \emph{arquitecte t\`{e}cnic}), we did not see evidence that it was critical to ordering preferences in our broad set of items. We chose not to include this null result in our revision, given the brevity of the paper, and the difficulty of validating our measure of concept-formability. 
However, we do include references to and discussion of the relevant literature. 

There remains the possibility that the operationalization of concept-formability that we gleaned from the literature was unable to measure the relevant dimension of meaning. A more theoretically neutral version of this hypothesis is that some interaction between the noun and adjective will determine how closely the adjective is placed to the noun. This interaction could be caused by concept-formation, differential subjectivity, or other factors. To test for an interaction between nouns and adjectives in determining preferences, we performed a nested linear model comparison. The models predicted naturalness ratings either by \textsc{adjective} (i.e., the adjective farthest from the noun) only, or by \textsc{adjective} together with its interaction with \textsc{noun} (i.e., the modified noun).	The model comparison revealed  that noun-specific naturalness did not explain any variance in ordering preference above and beyond adjective-level naturalness ($F(1,234) = 1.10, p < 0.15$). We now report the results of this analysis in our revision.

There were two adjective-noun pairs in our data with trends in the predicted direction: the naturalness ratings for \emph{hard} and \emph{soft} suggested a preference to occur closer to the noun \emph{cheese}. (Plausibly because hard and soft cheeses are compound concepts.) While these adjective-noun interactions do not survive correction for multiple comparisons in our statistical analysis, they do indicate that a different set of materials might reveal by-noun effects on ordering preference. We therefore followed up on this result by replicating our naturalness rating experiment with a new set of nouns chosen to maximize the probability of by-noun effects.
Compound concepts will be described using the two-word name, presumably yielding more occurrences of this bigram than would be expected from the unigram frequencies of the noun and adjective.
We thus chose new nouns whose co-occurrence probability with our 26 adjectives in the BNC is far greater than one would predict on the basis of their individual word probabilities: \emph{apple, cheese, eyes, hair}. We also included the noun \emph{thing} because it occurred naturalistically with the greatest number (23) of our 26 adjectives.
By taking all combinations of these nouns and our 26 adjectives we arrive at a set of items that we expect to include both compound concepts and ordinary modification.
We used these new materials in a direct replication of our naturalness ratings experiment, and performed the same nested model comparison predicting naturalness ratings either by \textsc{adjective} only, or by \textsc{adjective} together with its interaction with \textsc{noun}. The model comparison revealed that noun-specific ratings did not explain any additional variance in ordering preference beyond adjective-level ratings.

We also replicated our faultless disagreement subjectivity experiment with this new set of materials: faultless disagreement scores account for 84\% of the variance in the new naturalness ratings ($r^2$ 0.84, 95\% CI [0.64,  0.91]; cf.~the $r^2$ value of 0.88 from our original materials). Thus, while we fail to find evidence of noun-specific effects both in our original materials and in materials specifically designed to deliver such effects, we continue to see that subjectivity predicts adjective ordering preferences. We report on the details of this replication in a new supplement that we have added to our manuscript.

 %Thus, we fail to find evidence of noun-specific effects in our materials, but we leave open the possibility of finding them in a different set of materials. In our revision, we now discuss this result in our analysis of the preference data.



%[TODO: add noun analysis on preference data. speculate on materials that might show a bigger by-noun effect?]


%Since we have established that adjective subjectivity is a significant factor in determining ordering, we can at least partly test this hypothesis by looking at differences in subjectivity depending on which noun the adjective is combined with.
%%main determinant in ordering preferences is whether or not an adjective combines with a noun to form a new, complex concept, and if subjectivity estimates simply stand proxy for rates of concept-formability, then we might expect to find noun-specific effects in our data. 
%%Not all nouns are equally likely to form complex concepts with the adjectives that modify them, so our subjectivity measures and ordering preference ratings might vary depending on the specific nouns involved. 
%To test the role of the specific noun in predicting ordering preferences, we compared the predictions of our subjectivity measure with and without noun-specific information (i.e., faultless disagreement measures with and without averaging across the nouns that adjectives modified). In both cases, we used our subjectivity measure to predict noun-specific ordering preferences. A nested linear model comparison revealed that noun-specific subjectivity did not explain any variance in ordering preference above and beyond adjective-level subjectivity ($F(1,255) = 0.44, p < .51$); adjusted $R^2$ for both models was $0.7$. Thus, we fail to find evidence of noun-specific effects. In our revision, we now discuss this result in XXX.


\item  \emph{Subkind descriptions could be a confounding factor in the corpus study if color or material terms are more frequently used to create complex concepts.}

We evaluated the role of subjectivity in predicting the naturalness ratings data (Experiment 1 in the paper), not our corpus results (though the two are strongly correlated). Moreover, given the lack of evidence for the role of concept-formability we observed in our follow-up experiment (see the response to the previous point), we believe that there is little reason to suspect it to be a confounding factor in the findings we reported. 



\item \emph{Other known factors that affect adjective ordering like contrastiveness in discourse should be discussed.}

We now mention contrastiveness in discourse in our revision. As we discuss, contrastiveness follows straightforwardly from a kind of markedness implicature: marked forms (i.e., dis-preferred orderings) yield marked interpretations. The work lies in determining the preferred orderings from which these contrastive uses depart.
We also point out that many other situational factors are likely to influence ordering; it is the more general tendencies we are concerned with here. %[NOTE: add a sentence to that last point in manuscript.]


\item \emph{The finding that subjectivity predicts adjective ordering preferences lacks an explanation.}

We agree. However, as we discussed at the beginning of this letter, we do not believe the lack of explanation detracts from the significance of our findings: if anything, the puzzle of this empirical relationship increases the theoretical interest.


\item \emph{Linguists were already aware of subjectivity in adjectives.}

We agree, and it was not our intention to suggest otherwise: linguists and philosophers were aware that subjectivity is a theoretically interesting quantity. However, ours is the first study to systematically connect subjectivity with ordering tendencies. In our revision, we now mention previous work on adjective subjectivity---including the Kennedy paper that the reviewer recommended---in connection with our faultless disagreement task (see our response to Reviewer 2's first point below).

\item \emph{How was the number of participants chosen?}

The answer, as is often the case, starts with the senior author's intuition based on experience with many similar studies. In more detail: In a pilot experiment that we do not report in our paper, we ran a version of the ``subjectivity'' experiment in which 30 participants rated adjective subjectivity without any accompanying noun.\footnote{The full experiment is \href{http://web.stanford.edu/~scontras/adjective_ordering/experiments/6-subjectivity/subjectivity.html}{viewable online here}.} We found clear differences in subjectivity ratings across adjectives and classes, suggesting we had enough power in our analyses. 
Given the identical nature of the tasks, we ran the same number of participants in the subjectivity-noun experiment we report in the paper (Experiment 3). Indeed, the results of the two ``subjectivity'' experiments (i.e., with and without an accompanying noun) are extremely highly correlated ($r^2=0.97$; 95\% CI [0.95,  0.98]), hence serving as a replication of the result. We reported the results of the subjectivity-noun experiment in the paper to allow for a more straightforward comparison with the faultless disagreement task, which included modified nouns.

Without a similar pilot experiment with which to check our effect size, we ran ten more subjects in the faultless disagreement experiment (Experiment 4)---that is, we decided to err on the side of greater sensitivity. (As it turns out, this wasn't necessary: if we simulate what would have happened if we had run fewer participants---by sampling a random subset of 30 participants from the original 40 and using their responses to predict ordering preferences---we continue to see the success of subjectivity in predicting ordering preferences. The full faultless disagreement data set accounted for 88\% of the variance in the ordering preferences; on 100 random samples, the 30-participant subsets accounted for an average of 87\% of the variance ($r^2=0.87$; 95\% CI [0.76, 0.94]).)


\item \emph{Was any effort made to control how often adjectives appeared with other adjectives in the ordering experiment?}

The pairing of adjectives was completely random, with the exception that no adjectives from the same class were paired with each other. We opted for a random pairing as a simple-to-implement way to avoid by-item bias in our stimuli. (Fully counterbalancing the pairing would be difficult with so many items, and randomization tends to be sufficient with the sample sizes we were using. The use of modern regression analyses corrects for unequal sample sizes.) 

\een




\subsubsection*{Reviewer 2:}

\ben

\item \emph{There are two notions of faultless disagreement: 1) concerning semantic content, and 2) concerning context sensitivity and perspective. It was not clear which notion we intended.}

We were not sufficiently clear in our original discussion of faultless disagreement; our analysis only permits the more general notion that does not distinguish these senses---as the reviewer writes ``it may be that the kind of subjectivity that is relevant for adjective ordering is of a more general nature, and does not distinguish between subjective perspectives that are relevant for fixing semantic content in context and subjective perspectives that give rise to faultless disagreement in cases of sameness of semantic content.'' We now mention that the many factors likely affect subjectivity as measured by our faultless disagreement task, including specific semantic content and context sensitivity. Also, we now give references for faultless disagreement, so that readers may follow up on these distinctions. %We have also changed the terminology in our revision to reflect the distinction between the theory-specific notion of faultless disagreement and our use of the construct to evaluate adjective subjectivity. [TODO: are we going to change terminology or not?]

\item \emph{If successful referential communication is the driving factor, then we would predict the subjectivity gradient for post-nominal adjectives, but not for pre-nominal adjectives.}

This is an excellent point (indeed, one we have agonized over). From our original discussion of speaker utility, one might conclude that we endorse an explanation in terms of successful reference resolution during a linear parse of adjective-noun object descriptions. But as the reviewer rightly points out, a linear view of the phenomenon makes correct predictions only in the case of post-nominal languages. In our revision, we have clarified our conjecture: to the extent that ordering preferences emerge from on-line pressures of reference resolution, these pressures shape the hierarchical but not necessarily linear ordering of adjectives. We have also attempted to make it very clear that this is only a conjecture, and only a partial explanation.

\item \emph{What about numerals, which some people take to be adjectives?}

Adjectives are just one of many elements that may occur in complex nominal constructions. Other classes of elements include demonstratives and numerals. In his Universal 20, Greenberg (1963) observes that the relative order of these higher-order classes is also stable cross-linguistically (a recent \emph{PNAS} article used behavioral measures to support this universal; Culbertson and Adger, 2014).  We have added a brief discussion of this and other ordering preferences, which we take to evidence potential semantic constraints from composition. In particular, we take the distinct behavior of numerals to support their treatment as a class distinct from adjectives. However, we think our empirical methods may help in studies of ordering for other classes, and have indicated such in the revision.

\een




\noindent Thank you again for the thorough and thoughtful comments on our work. We hope that you will like the new version of the paper. Please let us know if you require additional information. We look forward to hearing from you!\\[25pt]


\noindent Yours sincerely,\\[10pt]

\noindent Gregory Scontras, Judith Degen, and Noah D.~Goodman



\end{document}















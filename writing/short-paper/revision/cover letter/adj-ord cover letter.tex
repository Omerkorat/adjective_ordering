\documentclass[12pt]{article}
\usepackage[hmargin={1in},vmargin={1in,1in},foot={.6in}]{geometry}   
\geometry{letterpaper}              
%\usepackage[parfill]{parskip}
\usepackage{color,graphicx}
\usepackage{setspace}
\usepackage{amsmath}
\usepackage{amssymb}
\usepackage{varioref}
\usepackage{textcomp}
\usepackage{avm}
\usepackage{textcomp}
\usepackage{mflogo}
\usepackage{wasysym}
\usepackage[normalem]{ulem}
\usepackage{hyperref}

\newcommand{\HRule}{\rule{\linewidth}{0.25mm}}

\usepackage{fancyhdr} % This should be set AFTER setting up the page geometry
\pagestyle{plain} % options: empty , plain , fancy
\lhead{}\chead{}\rhead{}
\renewcommand{\headrulewidth}{.5pt}
\lfoot{}\cfoot{\thepage}\rfoot{}
\newcommand{\txtp}{\textipa}
\renewcommand{\rm}{\textrm}
\newcommand{\sem}[1]{\mbox{$[\![$#1$]\!]$}}
\newcommand{\lam}{$\lambda$}
\newcommand{\lan}{$\langle$}
\newcommand{\ran}{$\rangle$}
\newcommand{\type}[1]{\ensuremath{\left \langle #1 \right \rangle }}

\newcommand{\bex}{\begin{exe}}
\newcommand{\eex}{\end{exe}}
\newcommand{\bit}{\begin{itemize}}
\newcommand{\eit}{\end{itemize}}
\newcommand{\ben}{\begin{enumerate}}
\newcommand{\een}{\end{enumerate}}

%\linespread{1.5}
\thispagestyle{plain}

\begin{document}

{\flushright

\vspace{25pt}
Gregory Scontras\\
Judith Degen\\
Noah D.~Goodman\\
Department of Psychology\\
Stanford University\\
Stanford, CA 94305\\[20pt]

\noindent October XXX, 2015\\[20pt]}


\noindent Dear Editor,\\

\noindent We would like to thank you and the two anonymous reviewers for the helpful reviews of our paper, ``Subjectivity predicts adjective ordering preferences.'' As you will recall, you and Reviewer 2 were in agreement that adjective ordering would be an excellent topic for a PNAS paper. However, both you and the reviewers expressed concern that our work does not move the inquiry forward far enough. We believe this concern stems from a failure on our part to do justice to the significant empirical contributions our paper makes.

Adjective ordering preferences continue to recur in discussions of language universals precisely because of their reported regularity within and across languages. However, even in the best studied case---English---the investigations to date have been largely impressionistic, rather than empirical. Most authors report their own intuitions or the intuitions of a handful of informants, using a highly constrained set of adjectives and nouns. The only large-scale empirical studies of ordering preferences are Martin 1969, who limits himself to behavioral measures (i.e., speaker intuitions), and Wulff 2013, who limits herself to corpus measures (i.e., relative frequencies). Our study marries these two approaches, using both behavioral measures and corpus analyses to arrive at clear estimates of the  preferences themselves.

There is even less empirical work on the factors that contribute to these preferences, which is where our paper makes its largest contribution. Most authors do not attempt to operationalize their hypotheses, let alone test them. Once again, Martin 1969 stands apart. However, of the four different aspects of adjective meaning that he hypothesizes predict ordering preferences, the best-performing measure is adjective ``definiteness;'' it accounts for 55\% of the variance in his preference data. Our study has more adjectives, 10 times more nouns, two different operationalizations of the subjectivity measure, and two different analyses for each operationalization. We show that at its \emph{worst}, subjectivity predicts 70\% of the variance in the ordering preference data; at its best, subjectivity predicts 88\% of the variance. Put simply, no other study comes close in its empirical rigor or success.

While we agree with Reviewer 1 that the finding that subjectivity predicts ordering preferences lacks an explanation, we disagree on its significance: our finding represents serious progress in the investigation of ordering preferences. With clear estimates of both the preferences themselves and the aspect of adjective meaning that determines them (i.e., adjective subjectivity), work on ordering preferences may finally move beyond informal descriptions to rigorous scientific inquiry. Moreover, the implications of this work are not limited to the relative order of adjectives. The most obvious extension is adverb ordering, which appears similarly robust across languages and similarly poorly understood. In our revision, we emphasize the significance of our findings and their potential extension to other cross-linguistic regularities.

In the remainder of this letter, we discuss in more detail how we responded to each of the reviewers' concerns.


\subsubsection*{Reviewer 1:}

\ben

\item \emph{The fundamental factor in predicting adjective ordering is whether an adjective is used to form a complex concept/subkind description or not.}

We considered it unlikely that a binary distinction like concept formation would be able to predict the gradience in our preference data. Still, we found the hypothesis very compelling, which is why we set out to test it. As with the studies in our paper, the work lies in operationalizing an abstract notion like whether or not an adjective forms a complex concept. The literature on the topic (McNally and Boleda, 2004; Svenonius, 2008) assumes the viability of this factor; the closest these papers come to proposing a predictive empirical measure of concept formability is the following distinction.

According to McNally and Boleda, the key issue is one of entailment. When an adjective modifies a noun intersectively, the objects described hold both the property named by the noun and the property named by the adjective: a ``male architect'' is both male and an architect. When an adjective and a noun combine to form a complex concept (i.e., a subkind description), the objects described hold the property named by the noun, but not necessarily the property named by the adjective; the modification is (ostensibly) subsective. The authors give the Catalan example \emph{arquitecte t\`{e}cnic} ``technical architect,'' which names architects but not necessarily technical things. The semantic analysis given to adjectives that form complex concepts requires them to compose first with nouns, before run-of-the-mill intersective adjectives; thus, the fundamental factor in predicting adjective ordering ought to be whether an adjective forms a complex concept.

Using our original set of materials, this is precisely what we tested: do the objects named by an adjective-noun description hold 1) the property named by the adjective, and 2) the property named by the noun;\footnote{The full experiment is \href{http://web.stanford.edu/~scontras/adjective_ordering/experiments/9-concept-formability/concept-formability.html}{viewable online here}.} then, does concept formability predict ordering preferences? We tested 40 participants on Mechanical Turk. The adjective ratings predict 8\% of the variance in our preference data (r$^{2}=0.08$; 95\% CI [0.00,  0.33]). The noun ratings predict 36\% of the variance (r$^{2}=0.36$; 95\% CI [0.07,  0.62]). Recall that at its worst, subjectivity predicts 70\% of the variance in our preference data.

We failed to find a role of concept formability in the determination of adjective ordering preferences. Given its null result, we saw no reason to include this experiment in our paper. However, in our revision, we included references to and discussion of the recommended papers.


\item  \emph{Subkind descriptions could be a confounding factor in the corpus study if color or material terms are more frequently used to create complex concepts.}

We evaluated the role of subjectivity in predicting the naturalness ratings data (Experiment 1 in the paper), not our corpus results. Moreover, given the lack of evidence for the role of concept-formability we observed in our follow-up experiment (see the response to the previous point), we believe that there is little reason to suspect it to be a confounding factor in the findings we reported.


\item \emph{Other known factors that affect adjective ordering like contrastiveness in discourse should be discussed.}

We now mention contrastiveness in discourse in our revision. As we discuss, contrastiveness follows straightforwardly from a kind of markedness implicature: marked forms (i.e., dis-preferred orderings) yield marked interpretations. The work lies in determining the preferred orderings from which these contrastive uses depart.


\item \emph{The finding that subjectivity predicts adjective ordering preferences lacks an explanation.}

Response.

\item \emph{Linguists were already aware of subjectivity in adjectives.}

Response.

\item \emph{How was the number of participants chosen?}

Response.

\item \emph{Was any effort made to control how often adjectives appeared with other adjectives in the ordering experiment?}

Response

\een




\subsubsection*{Reviewer 2:}

\ben

\item \emph{There are two notions of faultless disagreement: 1) concerning semantic content, and 2) concerning context sensitivity and perspective. It was not clear which notion we intended.}

Response

\item \emph{If successful referential communication is the driving factor, then we would predict the subjectivity gradient for post-nominal adjectives, but not for pre-nominal adjectives.}

Response.

\item \emph{What about numerals, which some people take to be adjectives?}

Response.

\een




\noindent Thank you again for the thorough and thoughtful comments on our work. We hope that you all like the new version of the paper. Please let us know if you require additional information. We look forward to hearing from you!\\[25pt]


\noindent Yours sincerely,\\[10pt]

\noindent Gregory Scontras, Judith Degen, and Noah D.~Goodman



\end{document}















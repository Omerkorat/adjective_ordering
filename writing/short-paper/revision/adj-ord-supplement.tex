\documentclass[12pt]{article}
\usepackage[hmargin={1in},vmargin={1in,1in},foot={.6in}]{geometry}   
\geometry{letterpaper}              
%\usepackage[parfill]{parskip}
\usepackage{color,graphicx}
\usepackage{setspace}
\usepackage{amsmath}
\usepackage{amssymb}
\usepackage{varioref}
\usepackage{textcomp}
%\usepackage{avm}
\usepackage{textcomp}
\usepackage{mflogo}
\usepackage{wasysym}
\usepackage[normalem]{ulem}
\usepackage{hyperref}

\newcommand{\HRule}{\rule{\linewidth}{0.25mm}}

\usepackage{fancyhdr} % This should be set AFTER setting up the page geometry
\pagestyle{plain} % options: empty , plain , fancy
\lhead{}\chead{}\rhead{}
\renewcommand{\headrulewidth}{.5pt}
\lfoot{}\cfoot{\thepage}\rfoot{}
\newcommand{\txtp}{\textipa}
\renewcommand{\rm}{\textrm}
\newcommand{\sem}[1]{\mbox{$[\![$#1$]\!]$}}
\newcommand{\lam}{$\lambda$}
\newcommand{\lan}{$\langle$}
\newcommand{\ran}{$\rangle$}
\newcommand{\type}[1]{\ensuremath{\left \langle #1 \right \rangle }}

\newcommand{\bex}{\begin{exe}}
\newcommand{\eex}{\end{exe}}
\newcommand{\bit}{\begin{itemize}}
\newcommand{\eit}{\end{itemize}}
\newcommand{\ben}{\begin{enumerate}}
\newcommand{\een}{\end{enumerate}}

%\linespread{1.5}
\thispagestyle{plain}

\title{Supporting information:\\ An excursus on noun effects}
\author{Gregory Scontras, Judith Degen, Noah D.~Goodman}
\date{}

\begin{document}

\maketitle


\emph{The fundamental factor in predicting adjective ordering is whether an adjective is used to form a complex concept/subkind description or not.}

We find this hypothesis intriguing---perhaps concept-formability indeed determines ordering preferences (and therefore correlates with subjectivity)?
%We considered it unlikely that a binary distinction like concept formation would be able to predict the gradience in our preference data. Still, we found the hypothesis very compelling, which is why 
We set out to test the hypothesis; as with the studies in our paper, the work lies in operationalizing an abstract notion like whether or not an adjective tends to form a complex concept. The literature on the topic (McNally and Boleda, 2004; Svenonius, 2008) presupposes that intuitions about concept formability are systematic and generalizable; the closest we found in these papers to a proposal for an empirical measure of this factor is the following distinction.

According to McNally and Boleda, the key issue is one of entailment. When an adjective modifies a noun intersectively, the objects described hold both the property named by the noun and the property named by the adjective: a ``male architect'' is both male and an architect (McNally and Boleda, 2004:179, ex.~2). When an adjective and a noun combine to form a complex concept (i.e., a subkind description), the objects described hold the property named by the noun, but not necessarily the property named by the adjective; the modification is (ostensibly) subsective. The authors give the Catalan example \emph{arquitecte t\`{e}cnic} ``technical architect,'' which names architects but not necessarily technical things (McNally and Boleda, 2004:179, ex.~1; cf.~the discussion of \emph{wild rice} in Svenonius 2008). 
%
Using our original set of materials, we tested whether the objects named by an adjective-noun description hold 1) the property named by the adjective, and 2) the property named by the noun.\footnote{The full experiment is \href{http://web.stanford.edu/~scontras/adjective_ordering/experiments/9-concept-formability/concept-formability.html}{viewable online here}.} We tested 40 participants on Mechanical Turk. 


The semantic analysis given by these authors to adjectives that form complex concepts requires them to compose first with nouns, before run-of-the-mill intersective adjectives; thus, the fundamental factor in predicting adjective ordering ought to be whether an adjective forms a complex concept. Does concept-formability predict ordering preferences?
The adjective concept-formability ratings predict 8\% of the variance in our preference data (r$^{2}=0.08$; 95\% CI [0.00,  0.33]). The noun ratings predict 36\% of the variance (r$^{2}=0.36$; 95\% CI [0.07,  0.62]). Recall that at its worst, subjectivity predicts 70\% of the variance in our preference data.
While it is quite possible that concept-formability plays an important role for some cases (such as \emph{arquitecte t\`{e}cnic}), we did not see evidence that it was critical to ordering preferences in our broad set of items. We chose not to include this null result in our revision, given the brevity of the paper. 
However, we do include references to and discussion of the relevant literature. 

There remains the possibility that the operationalization of concept-formability that we gleaned from the literature was unable to measure the relevant dimension of meaning. A more theoretically neutral version of this hypothesis is that some interaction between the noun and adjective will determine how closely the adjective is placed to the noun. This interaction could be caused by concept-formation, differential subjectivity, or other factors. To test for an interaction between nouns and adjectives in determining preferences, we performed a nested linear model comparison. The models predicted naturalness ratings either by \textsc{adjective} (i.e., the adjective farthest from the noun) only, or by \textsc{adjective} together with its interaction with \textsc{noun} (i.e., the modified noun).	The model comparison revealed  that noun-specific naturalness did not explain any variance in ordering preference above and beyond adjective-level naturalness ($F(1,234) = 1.10, p < 0.15$). 

There were two adjective-noun pairs in our data with trends in the predicted direction: the naturalness ratings for \emph{hard} and \emph{soft} suggested a preference to occur closer to the noun \emph{cheese}. (Plausibly because hard and soft cheeses are compound concepts.) While these adjective-noun interactions do not survive correction for multiple comparisons in our statistical analysis, they do indicate that a different set of materials might---indeed, probably would---reveal by-noun effects on ordering preference.  Thus, we fail to find evidence of noun-specific effects in our materials, but we leave open the possibility of finding them in a different set of materials. In our revision, we now discuss this result in our analysis of the preference data.



\emph{Subkind descriptions could be a confounding factor in the corpus study if color or material terms are more frequently used to create complex concepts.}

We evaluated the role of subjectivity in predicting the naturalness ratings data (Experiment 1 in the paper), not our corpus results (though the two are strongly correlated). Moreover, given the lack of evidence for the role of concept-formability we observed in our follow-up experiment (see the response to the previous point), we believe that there is little reason to suspect it to be a confounding factor in the findings we reported. 





\end{document}














